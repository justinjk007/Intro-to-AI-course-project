% Created 2018-04-10 Tue 13:17
% Intended LaTeX compiler: pdflatex
\documentclass[a4paper,12pt]{article}
\usepackage[utf8]{inputenc}
\usepackage[T1]{fontenc}
\usepackage{graphicx}
\usepackage{grffile}
\usepackage{longtable}
\usepackage{wrapfig}
\usepackage{rotating}
\usepackage[normalem]{ulem}
\usepackage{amsmath}
\usepackage{textcomp}
\usepackage{amssymb}
\usepackage{capt-of}
\usepackage{hyperref}
\usepackage[T1]{fontenc} % For times new roman font
\usepackage{mathptmx} % For times new roman font
\linespread{1.3} % Change line spacing
\usepackage{xcolor}
\usepackage{soul}
\usepackage{helvet}
\usepackage{listings}
\usepackage{inconsolata}
\usepackage{xcolor-solarized}
\definecolor{foreground}{RGB}{184, 83, 83} % For verbatim
\definecolor{background}{RGB}{255, 231, 231} % For verbatim
\let\OldTexttt\texttt
\renewcommand{\texttt}[1]{\OldTexttt{\footnotesize\colorbox{background}{\textcolor{foreground}{#1}}}}
\newenvironment{helvetica}{\fontfamily{phv}\selectfont}{\par}
\usepackage{hyperref} % Make the hyper-links prettier
\hypersetup{
colorlinks=true,
linkcolor=blue!70!white,
urlcolor=blue!95!black
}
\usepackage{enumitem}
\setlist[1]{itemsep=5pt}
\lstdefinelanguage{cpp}{
language=C++,
morekeywords={cerr,exit,string},
deletekeywords={...},
escapeinside={\%*}{*)},
showspaces=false,
showstringspaces=false,
showtabs=false,
stepnumber=1,
tabsize=4,
breakatwhitespace=false,
breaklines=true,
backgroundcolor=\color{solarized-base3},
basicstyle=\scriptsize\ttfamily\color{solarized-base0},
commentstyle=\itshape\color{solarized-base01},
keywordstyle=\color{solarized-green},
identifierstyle=\color{solarized-blue},
stringstyle=\color{solarized-cyan},
moredelim = *[l][\color{solarized-orange}]{\#},
moredelim = **[s][\color{solarized-cyan}]{<}{>},
rulecolor=\color{black},
literate={{\%d}}{{\textcolor{solarized-red}{\%d}}}2
{{\%2d}}{{\textcolor{solarized-red}{\%2d}}}3
{{\\n}}{{\textcolor{solarized-red}{\textbackslash{}n}}}2,
}
\author{Justin Kaipada}
\date{\today}
\title{}
\hypersetup{
 pdfauthor={Justin Kaipada},
 pdftitle={},
 pdfkeywords={},
 pdfsubject={},
 pdfcreator={Emacs 25.3.1 (Org mode 9.1.6)},
 pdflang={English}}
\begin{document}


\section{AI-Course-Project}
\label{sec:orgb31c682}

We have implemented 3 scenarios of agent based search environment where each agent has to collect 5
of its targets spawned randomly in the environment with a fixed location.

\begin{itemize}
\item Scenario 1: \textbf{Competition:} Here only public broadcast is allowed. Each agent may lie or
broadcast the correct information about other agent's targets into a public channel that every
agent can see. An iteration is over as soon as one agent collect all of its targets.
\item Scenario 2: \textbf{Collaboration:} Here public and private broadcast is allowed. An agent can
broadcast information publicly or specificly to an agents channel. Here we may or may not lie in
the public channel and always tell the truth in private communication. Iteration is over when all
the agents collects all the targets.
\item Scenario 3: \textbf{Compassionate:} This is the same as Collaboration except no one will ever lie and
the game is over when one agent collect all of its targets.
\end{itemize}

The Executables can be found in the \texttt{bin} directory

\section{Setup}
\label{sec:org078fa71}
\subsection{SDL2 is the main rendering library used}
\label{sec:org99d68cc}

\textbf{SDL2image} add-on is also used to load add-ons

\subsubsection{On Windows}
\label{sec:orgeaffb0c}
\begin{enumerate}
\item Setup SDL2
\label{sec:org325805b}
\begin{itemize}
\item Download \textbf{SDL2} libs from \href{https://www.libsdl.org/download-2.0.php}{here}
\item Specifically you need the Developmental libs.
\item I am using Visual Studio right now and I downloaded \texttt{SDL2-devel-2.0.8-VC.zip}
\item Extract it somewhere, I extracted it to \texttt{C:\textbackslash{}Dev\textbackslash{}SDL2-2.0.8}
\item Create a new file named \texttt{sdl2-config.cmake} in there with the following content
\end{itemize}

\lstset{language=cmake,label= ,caption= ,captionpos=b,numbers=none,language=cpp}
\begin{lstlisting}
set(SDL2_INCLUDE_DIRS "${CMAKE_CURRENT_LIST_DIR}/include")
# Support both 32 and 64 bit builds
if (${CMAKE_SIZEOF_VOID_P} MATCHES 8)
  set(SDL2_LIBRARIES "${CMAKE_CURRENT_LIST_DIR}/lib/x64/SDL2.lib;${CMAKE_CURRENT_LIST_DIR}/lib/x64/SDL2main.lib")
else ()
  set(SDL2_LIBRARIES "${CMAKE_CURRENT_LIST_DIR}/lib/x86/SDL2.lib;${CMAKE_CURRENT_LIST_DIR}/lib/x86/SDL2main.lib")
endif ()
string(STRIP "${SDL2_LIBRARIES}" SDL2_LIBRARIES)
\end{lstlisting}

\begin{itemize}
\item Now open control panel to \texttt{Edit environment variables} , you can just search for this term too
\item Add a new \textbf{System variable} named \texttt{SDL2\_DIR} and set the value to the directory
where you extracted the lib for me it will be \texttt{C:\textbackslash{}Dev\textbackslash{}SDL2-2.0.8}.
\item I will be using \textbf{x64} libs so I will also edit the \texttt{Path} variables for
the user and add \texttt{C:\textbackslash{}Dev\textbackslash{}SDL2-2.0.8\textbackslash{}lib\textbackslash{}x64} folder to path.
\end{itemize}

\item Setup SDL2image
\label{sec:org3c73054}
\begin{itemize}
\item Download \textbf{SDL2image} libs from \href{https://www.libsdl.org/projects/SDL\_image/}{here}
\item Specifically you need the Developmental libs.
\item I am using Visual Studio right now and I downloaded \texttt{SDL2\_image-devel-2.0.3-VC.zip}
\item Extract it somewhere, I extracted it to \texttt{C:\textbackslash{}Dev\textbackslash{}SDL2\_image-2.0.3}
\item Create a new file named \texttt{sdl2\_image-config.cmake} in there with the following content.
\end{itemize}

\lstset{language=cmake,label= ,caption= ,captionpos=b,numbers=none,language=cpp}
\begin{lstlisting}
#+BEGIN_SRC cmake
set(SDL2_IMAGE_INCLUDE_DIRS "${CMAKE_CURRENT_LIST_DIR}/include")
# Support both 32 and 64 bit builds
if (${CMAKE_SIZEOF_VOID_P} MATCHES 8)
  set(SDL2_IMAGE_LIBRARIES "${CMAKE_CURRENT_LIST_DIR}/lib/x64/SDL2_image.lib")
else ()
  set(SDL2_IMAGE_LIBRARIES "${CMAKE_CURRENT_LIST_DIR}/lib/x86/SDL2_image.lib")
endif ()
string(STRIP "${SDL2_IMAGE_LIBRARIES}" SDL2_IMAGE_LIBRARIES)
\end{lstlisting}

\begin{itemize}
\item Now open control panel to \texttt{Edit environment variables}
\item Add a new \textbf{System variable} named \texttt{SDL2\_image\_DIR} and set the value to the directory
where you extracted the lib for me it will be \texttt{C:\textbackslash{}Dev\textbackslash{}SDL2\_image-2.0.3}
\item I will be using \textbf{x64} libs so I will also edit the \texttt{Path} variables for
the user and add \texttt{C:\textbackslash{}Dev\textbackslash{}SDL2\_image-2.0.3\textbackslash{}lib\textbackslash{}x64} folder to path.
\end{itemize}


\begin{itemize}
\item \textbf{Restart the system so the environment variables are set and the path re-read}.
\item Now you can build the code with CMake
\end{itemize}
\end{enumerate}

\subsubsection{On Linux(Ubuntu)}
\label{sec:orga96cd07}

Download both \texttt{SDL2} and \texttt{SDL2\_image} like this
\lstset{language=sh,label= ,caption= ,captionpos=b,numbers=none,language=cpp}
\begin{lstlisting}
sudo apt install libsdl2-dev
sudo apt install libsdl2-image-dev
\end{lstlisting}

The \textbf{CMake} file I have can automatically find and build these libs

\subsection{Qt is used to manage front-end and back-end threads and their communication}
\label{sec:orgb528a9c}
\subsubsection{On Windows}
\label{sec:org05ce9b9}
\begin{itemize}
\item Download qt libs from here \url{https://www.qt.io/download}
\item Add the library directory to path as \texttt{QT\_DIR}, for me it is \texttt{C:\textbackslash{}Dev\textbackslash{}Qt\textbackslash{}5.9.1}
\item Also add the library binaries to \texttt{Path} for me it is \texttt{C:\textbackslash{}Dev\textbackslash{}Qt\textbackslash{}5.9.1\textbackslash{}msvc2015\_64\textbackslash{}bin}
\end{itemize}

\subsubsection{On Linux(Ubuntu)}
\label{sec:orgc5fc946}

\begin{itemize}
\item Install necessary Qt modules from an updated \textbf{PPA} like this one: \url{https://launchpad.net/\~beineri/+archive/ubuntu/opt-qt592-trusty}
\item Install necessary modules with these commands after adding the \textbf{PPA}
\lstset{language=sh,label= ,caption= ,captionpos=b,numbers=none,language=cpp}
\begin{lstlisting}
sudo apt install qt59base
sudo apt install qt59declarative
\end{lstlisting}
\end{itemize}

\subsection{Now you should be able to use CMake to build the source code}
\label{sec:org94cd16d}
\end{document}